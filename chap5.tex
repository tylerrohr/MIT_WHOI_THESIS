\chapter{Southern Ocean Iron Fertilization: An argument against commercialization but for continued research amidst lingering uncertainty.
}
\label{chap:5}
\raggedbottom

\clearpage



\section{Abstract}

In light of the challenges impeding substantive global action on climate change mitigation, some have begun to look at geoengineering as a viable alternative. Ocean Iron Fertilization (OIF) is one such strategy that seeks to increase oceanic drawdown of CO2 by stimulating marine phytoplankton growth in large iron-limited swaths of the Southern Ocean. Unfortunately, there are serious doubts that a global scale, iron-induced sequestration pathway could be sustainable, meaningful, additional, permanent, free of leakage, and absent of adverse side effects. While reduced uncertainty could one day reveal a reasonable, measured, approach to leverage OIF under unilateral authority and dynamic management, commercialization is, and will remain, unviable. Measurement challenges, unreliable auditing, ambiguous baselines compromised by high frequency variability, poor enforcement, and severe externalities, would together cripple a market-based approach to implementation and should preclude adoption into emerging compliance offset markets. Unfortunately, commercial ventures on voluntary offset markets motivated by the relative ease of OIF, market incentives and a loose regulatory framework have and may continue to emerge contrary to the scientific consensus. Over time, further research may or may not validate OIF as a justifiable risk, it will, however, almost certainly help delegitimize the reckless commercialization of OIF and dismiss the notion that the market can constrain such a complex and non-linear system without serious consequences.


\section{Introduction/Background}

Although geophysical scientists can say with increasing certainty that anthropogenic climate change poses a serious threat to human life and livelihood \parencite{ClimateChange20142014}, decisive political efforts to reduce emissions have proven slow and insufficient \parencite{ClarkHasKyotoprotocol2012,HoviImplementingLongTermClimate2009, ShearTrumpWillWithdraw2018}. Hedging against the challenges of a collective action crisis of this scale \parencite{ThompsonManagementAnarchyInternational2006} many have turned to adaptation to confront a future potentially plagued by unchecked fossil fuel consumption. The most dramatic of these adaptation strategies, known broadly as geoengineering, can be categorized as any deliberate, large-scale, manipulation of natural processes to effect the climate system, ostensibly to curb the effects of global warming. 

Geoengineering is an understandably enticing proposition. By targeting the symptoms of climate change for pennies on the dollar, geoengineering evades the daunting economic and cultural sacrifices that serious mitigation might demand. While its proponents preach the need to leverage human ingenuity \parencite{LynasGodSpeciesSaving2011}, the temptation of such a convenient deus ex machina raises some serious red flags. Critics stress that in the complex, non-linear earth system the treatment may not work as advertised, allowing the underlying cause to aggravate while unpredictable, side effects threaten to fester \parencite{LynasGodSpeciesSaving2011,Robock20reasonswhy2008}. Despite the controversy, the promise of a silver-bullet techno-fix has proven difficult to ignore, attracting attention from activists \parencite{KeithCaseClimateEngineering2013}, policy makers \parencite{FullCommitteeHearing2009,ReichleWORKINGPAPERCARBON1999}, and scientists alike \parencite{YoonOceanIronFertilization2016}. 

One of the most prominent geoengineering strategies is carbon sequestration. The global carbon cycle, which influences climate via the radiative capacity of atmospheric $CO_2$, largely operates on two time scales. The ‘slow’ carbon cycle, driven by tectonic activity, is balanced by volcanic outgassing and geological weathering \parencite{BernerAtmosphericcarbondioxide1990}. The ‘fast’ carbon cycle, driven by biology, is balanced by the reduction of inorganic carbon by photosynthesis and the oxidation of organic carbon by respiration \parencite{RiebeekCarbonCycleFeature2011}. Although these two cycles are naturally linked by the eventual fossilization of organic matter, the rapid anthropogenic consumption of fossil fuels is pumping CO2 out of the ‘slow’ cycle much faster than natural cycles can compensate for \parencite{ChisholmDisCreditingOceanFertilization2001}. The goal of carbon sequestration then is to geoengineer a carbon sink that is capable of routing carbon back into the ‘slow’ carbon cycle at a rate consummate with which we are extracting it. 

Ocean Iron Fertilization (OIF) is one of several carbon sequestration strategies. OIF seeks to harness the power of ocean biogeochemistry to amplify atmospheric $CO_2$ drawdown by augmenting inefficiencies in phytoplankton productivity. Microscopic marine phytoplankton account for roughly half of the photosynthetic carbon fixation on earth \parencite{FalkowskiGlobalCarbonCycle2000} and thanks to the rapid turnover time of their population ($\sim$1 week) they are able to influence climate on a much faster timescale \parencite{Falkowskioceaninvisibleforest2002} than terrestrial plants. Although most of the carbon fixed by marine photosynthesis is rapidly recycled and released back to the atmosphere, a small fraction ($\sim$ 15\%) sinks deep into the ocean where it might remain sequestered for tens to hundreds of years \parencite{LawsTemperatureeffectsexport2000}; This process is know as the biological pump \parencite{delaRochaChapterBiologicalPump2006}. OIF hopes to increase the strength and efficiency of the biological pump thereby stimulating greater net atmospheric $CO_2$ drawdown into the ocean.  

Variability in the biological pump is widely accepted as an important factor in regulating glacial-interglacial cycles in atmospheric $CO_2$ \parencite{BernermodelatmosphericCO1991, SigmanGlacialinterglacialvariations2000}, however the precise mechanisms that drive changes to the biological pump are less clear \parencite{FalkowskiEvolutionnitrogencycle1997}. In the late 1980s John Martin proposed that iron was limiting primary production over large swaths of the ocean \parencite{MartinIrondeficiencylimits1988}. Iron is a micronutrient essential to phytoplankton growth but only required in small concentrations relative to macronutrients like Nitrogen and Phosphorus. Regions, known as HNLCs (High Nitrate, Low Chlorophyll), that yield low primary productivity despite an abundance of unutilized macronutrients are now thought to iron-limited. Martin went on to hypothesize that variability in the airborne deposition of iron over HNLC regions has triggered variability in primary productivity, the biological pump, and the net drawdown of atmospheric $CO_2$  over glacial-interglacial cycles \parencite{MartinGlacialinterglacialCO2change1990}. Following the Martin Hypothesis, OIF seeks to fertilize HNLCs with the deliberate addition of iron in order to increase the strength of the biological pump and enhance oceanic drawdown of atmospheric $CO_2$.

Research over the ensuing decades ranging from shipboard in-vitro incubations \parencite{Martincaseiron1991} to mesoscale in-situ fertilization experiments \parencite{YoonOceanIronFertilization2016} has largely confirmed that iron fertilization can in fact stimulate primary production in HNLC regions. It is decidedly less clear, however, whether or not that increased productivity is routed into even localized export\parencite{BoydMesoscaleIronEnrichment2007}. When considered at a global scale, there remains a great deal of doubt over the validity of a sustainable iron induced sequestration pathway \parencite{WincklerOceandynamicsnot2016}. Additional concerns over the practical challenges of creating a credible auditing framework and the potential for unpredictable, adverse side effects create further complications.

Nevertheless, the lure of OIF remains theoretically seductive. HNLCs cover roughly a thirds of the world’s oceans \parencite{BoydMesoscaleIronEnrichment2007} and if continuous fertilization was generously assumed to utilize and export all previously unutilized nutrients, early models predicted an atmospheric drawdown as high as 50-100 ppm $CO_2$ \parencite{AumontGlobalizingresultsocean2006}. Despite problematical assumptions clouding these model estimates, geochemical proxies in ice cores \parencite{WincklerCovariantGlacialInterglacialDust2008} linking elevated iron to depressed atmospheric CO2 and recent observations of localized export after an artificially induced bloom \parencite{SmetacekDeepcarbonexport2012} have provided hope for proponents of OIF. If optimistic estimates for sequestration were even approached, OIF could potentially account for one of Pacala and Socolow’s twelve proposed stabilization wedges \parencite{PacalaStabilizationWedgesSolving2004}, enough to garner considerable attention \parencite{BuesselerExploringOceanIron2008}.

Amidst lingering uncertainty clouding the implementation, risk, and feasibility of OIF two predominant questions emerge. First, given the scope of the climate crisis could the commercialization of OIF under emerging compliance offset markets be a valuable piece of the solution? Second, is it prudent to dedicate valuable scientific resources to continuing incremental research targeted at reducing the uncertainty surrounding OIF? In light of looming challenges to the safety and efficacy of global-scale market based OIF, I present the case against commercialization in \textbf{Section 2}. In \textbf{Section 3} I argue, however, that even if commercialization under compliance markets is deemed unacceptable by scientists and policymakers, continued research is vital to delegitimize future developments in voluntary carbon markets that, at present, may be incentivized to proceed contrary to the precautionary principle.

\section{The case against commercialization}

At a bare minimum, the commercialization of OIF can not be advisable unless it satisfies the standards established by existing international offset markets. The Clean Development Mechanism \parencite{GillenwaterCleanDevelopmentMechanism2011}, developed under the Kyoto Protocol, has become the preeminent international offset program and has set basic guidelines for qualifying projects. Eligible projects must be, amongst other stipulations, permanent, additional, free of leakage, and amenable to monitoring \parencite{GillenwaterCleanDevelopmentMechanism2011}. In the context of OIF this amounts to two basic questions. First, does it work? That is, will fertilizing iron-limited swaths of the ocean stimulate new (“additional”) production that will sequester carbon into the deep ocean for long periods (“permanent”) without leading to downstream reductions in productivity (“leakage”) triggered by upstream macro-nutrient utilization (\textbf{Section 2.1})? Second, can it be measured? That is, can the net additional carbon sequestration from an individual OIF project be quantified accurately enough to ensure fair and consistent compensation (\textbf{Section 2.2})? Finally, although not explicitly mentioned in current market frameworks, a critical third question is implicit in any geoengineering project of this scale. Is it safe from adverse, unpredictable side effects (Section 2.3)? The litany of challenges, risks, and lingering uncertainty that beguile each of these questions is detailed respectively in \textbf{Sections 2.1}, \textbf{2.2} and \textbf{2.3}. In Section \textbf{2.4} I conclude that safe, predictable, and effective management of OIF on emerging offset markets is simply untenable. 

\subsection{Will it work?}

%%%%%%%%%%%%%%
\subsubsection{Does iron fertilization stimulate new production?}

After Martin’s Iron Hypothesis was proposed, early shipboard incubations \parencite{Martincaseiron1991} began to provide compelling evidence for iron limitation. Although these incubations were plagued by methodical biases, the results were enough to prompt an era of large scale, in-situ, iron enrichment experiments \parencite{ChisholmWhatControlsPhytoplankton1991}. Since 1990, 13 mesoscale artificial enrichment experiments have been performed, with 7 located in the Southern Ocean. 

These experiments were all carried out in a similar manner. Iron, dissolved into acidified seawater, was dumped into HNLC surface waters, while the concurrent addition of a $SH_6$, a biologically inert chemical tracer, allowed the water parcel to be traced and observed in a Lagrangian framework for the following 10-40 days \parencite{BoydMesoscaleIronEnrichment2007}. The ensuing biogeochemical response consistently showed an increase in photosynthetic efficiency, chlorophyll concentration, primary production and a corresponding drawdown of $pCO_2$ and macro-nutrients \parencite{YoonOceanIronFertilization2016}. Collectively, combined with observations of naturally enriched waters \parencite{deBaarSynthesisironfertilization2005}, results have demonstrated fairly unequivocally that iron fertilization does stimulate local productivity in certain HNLC regions \parencite{YoonOceanIronFertilization2016, BoydMesoscaleIronEnrichment2007}. 

Note, however, that the timing and location of these relatively small-scale experiments was carefully chosen. Blooms can still be limited, or co-limited, by light, silicate, macro-nutrients and grazing, leading to variability in the efficiency of iron fertilization. Discrepancies in local environmental, physical and seasonal conditions have been shown to have a pronounced effect on the relative success of any given particular patch fertilization \parencite{BoydMesoscaleIronEnrichment2007}. This means that to achieve even conservative estimates of stimulated growth, fertilization sites would need to be chosen with care and precision. In a dynamic ocean environment, this might not be practical, or possible, at the global scale. 

%%%%%%%%%%%%%%
\subsubsection{Is stimulated primary production routed into export production?}

Even if mass fertilization is scalable, it is critical to remember that iron limitation is only the first step of the Iron Hypothesis \parencite{BuesselerOceanIronFertilization2008}. Leveraging this insight into a sustainable global carbon sequestration program hinges largely on ambiguity over the long term fate of the new organic matter that is produced.

The stimulation of new production is only relevant to the medium-to-long term global carbon cycle if it is in turn sequestered for significant time scales. Most organic matter is however rapidly remineralized as it is respired in the surface ocean. This carbon is returned to the pCO2 pool and able to re-equilibrate with the atmosphere. For OIF to be a viable sequestration pathway a significant fraction of stimulated production must instead be exported out of the surface ocean to depth. 

Some naturally fertilized systems have been observed to yield reasonably strong export fluxes relative to nearby iron depleted water\parencite{BlainEffectnaturaliron2007, PollardSouthernOceandeepwater2009}, however, it is problematic to extrapolate natural laboratories to large scale geoengineering efforts, or even patch fertilizations. Only EIFEX \parencite{SmetacekDeepcarbonexport2012}, one of 13 total mesoscale enrichment experiments, actually observed an increase in export \parencite{YoonOceanIronFertilization2016}. Localized naturally fertilized regions represent a highly specific response to a particular set of environmental conditions that may not be easy or practical to replicate across the Southern Ocean \parencite{SalterDiatomrestingspore2012}. For instance, in natural systems iron is generally slowly and continuously supplied throughout the year, whereas in artificial fertilizations iron is deposited in pulsed rapid inputs. This results in a substantial fraction of the deposited iron being lost to abiotic processes such as particle scavenging prior to biological uptake \parencite{Bowiefateaddediron2001}. 

Even if the results from EIFEX can be replicated at scale, recent modeling work has predicted an underwhelming ceiling for integrated export in a continuously fertilized ocean compared to earlier studies \parencite{AumontGlobalizingresultsocean2006}. Current model estimates suggest that HNLCs are not capable of exporting any more than several hundred million tons of $CO_2$ per year \parencite{BuesselerOceanIronFertilization2008}, a stark contrast to the roughly 10 billion tons humanity releases annually \parencite{HumanNaturalDrivers2007}. 

%%%%%%%%%%%%%%
\subsubsection{Does export translate into long term storage?}

Regardless of export efficiency out of the surface ocean, only a very small fraction ($\sim<1\%$ of total export \parencite{PrenticeIPCCClimateChange2001}) of sinking organic matter will make it to the sediments where it can remain sequestered on geological time scales. The rest is remineralized at depth and eventually transported back to the surface ocean where it can be released into the atmosphere. The timescales over which this occurs vary with the remineralization depth \parencite{GnanadesikanEffectspatchyocean2003}, but generally fall on the order of 10-100s of years. Geoengineering advocates contend that if a large enough export flux is achieved, then this is long enough to buy time until other longer term climate solutions are developed. Questions remain, however, if this relatively mild best case scenario is even attainable.

During EIFEX, the lone artificial OIF experiment that found evidence of increased export, \textcite{SmetacekDeepcarbonexport2012} concluded that over half of the stimulated bloom’s biomass sank below 1000 meters. These results, however, must be considered in a broader context. The efficiency of the biological pump is highly variable on a regional and seasonal basis. On a seasonal basis, deep winter mixing which can penetrate hundreds of meters below the surface, could quickly return carbon remineralized relatively deep to the surface ocean. Regionally, targeting areas of deep water formation may increase the reminerialization depth as surface waters subduct, but may also introduce operational hazards from working in the coastal, heavily ice covered regions where deep water typically forms.

Further, some have pointed out that the export flux and variability in the remineralization depth can not alone describe oceanic carbon storage, pointing instead to preformed nutrient budgets which are additionally controlled by stoichiometry and circulation \parencite{GnanadesikanExportnotenough2008}. In this context, much larger space and time scales must be considered to quantify carbon storage, dramatically complicating the scalability and extrapolation of localized enrichment experiments.  


%%%%%%%%%%%%%%
\subsubsection{Do secondary effects on ecosystem structure effect the net sequestration of carbon?}

Southern Ocean iron enrichment experiments tend to preferentially increase the growth rates of diatoms, shifting community composition from smaller phytoplankton functional types to larger, silicate shelled, chain forming, diatom assemblages \parencite{HutchinsIronlimiteddiatomgrowth1998,HoffmannDifferentreactionsSouthern2006}. The stimulation of these rapidly sinking, heavy assemblages is key to increasing the efficiency of the biological pump \parencite{HoffmannDifferentreactionsSouthern2006} but might additionally trigger less desirable secondary consequences.  

Many of the heterotrophic zooplankton, such as copepods, that preferentially graze on diatoms form calcium carbonate ($CaCO_3$) shells \parencite{TsudaEvidencegrazinghypothesis2007}. The precipitation of $CaCO_3$ triggers a change in the speciation of the equilibrated carbonate system which leads to an increase in $pCO_2$ \parencite{FrankignoulleMarinecalcificationsource1994}. Although the dissolution of CaCO3 is in turn an effective sink for $pCO_2$, if it is first transported to depth calcification can act as a CO2 source on time scales of 10-100s of years. This process is known as the carbonate pump and has been observed to reduce the carbon sequestration capacity of natural systems by as much as 30\% \parencite{SalterCarbonatecounterpump2014}. If a stimulated diatom population preferentially favors calcifying grazers, indirect stimulation of the carbonate pump ($CO_2$ source) must be weighed against stimulation the biological pump ($CO_2$ sink).

More generally, increased primary production may not dictate a proportionate response in export production. Population size is not only regulated by bottom-up controls on phytoplankton growth rates such as light and nutrient (e.g. iron) limitation, but also by top-down controls on phytoplankton loss rates imposed predominately by grazers \parencite{BehrenfeldAnnualcyclesecological2013}. If highly stimulated growth rates improve grazing efficiency, a negative feedback loop could emerge in which increasing grazing rates damp population gains \parencite{RohrVariabilitymechanismscontrolling2017}. As carbon is transferred up the food chain, and partially respired along the way, increased grazing would in turn lead to less efficient carbon export \parencite{BoydImpactClimateChange2003}.


%%%%%%%%%%%%%%
\subsubsection{Does local macro-nutrient utilization compromise downstream productivity?}


Even if long term, local sequestration was successful, in a highly interconnected global ocean any local perturbation must be considered in the context of its ensuing downstream, non-local effects. By design, successful OIF would drawdown previously unused macro-nutrients and trap them at depth. While these unutilized macronutrients were previously of little use in iron limited surface waters, the depression of the nutrient profile leaves intermediate waters with reduced nutrient concentrations as well. These intermediate waters are eventually advected and upwelled at lower latitudes. To understand the net effect of Southern Ocean OIF we must also understand the price of reducing the supply of downstream macro-nutrients. That is, what is the cost of leakage?

Unfortunately, this question transcends the scope of modern observational capabilities. Several modelling studies, however, have suggested that leakage could be considerable, leading to a significant downstream reduction in primary production, atmospheric $CO_2$ drawdown and export production in the tropics \parencite{GnanadesikanEffectspatchyocean2003, AumontGlobalizingresultsocean2006, OschliesSideeffectsaccounting2010, SarmientoThreedimensionalsimulationsimpact1991}. Specifically, \textcite{OschliesSideeffectsaccounting2010} found that when integrated over 100 years, increased non-local outgassing compromised the net $CO_2$ drawdown by 20\%. Worse, Gnanadesikan and Sarmiento42 found that the integrated non-local reduction in export was 30 times greater than locally stimulated export. In turn, after 100 years only 2-44\% of the initially stimulated local export remained removed from the atmosphere. While non-local productivity could be damped for hundreds of years, the majority of artificially added iron is likely to be rapidly removed from the water column and buried in the sediments via particle scavenging \parencite{AumontGlobalizingresultsocean2006}, meaning the local stimulus could be short-lived relative to the non-local ramifications.


%%%%%%%%%%%%%%
\subsection{Auditing: Can it be measured? 
}

A robust auditing framework is a prerequisite for any reputable OIF plan and hinges largely on the ability to establish accurate and reliable estimates of sequestration. This is challenging for any carbon sequestration scheme but is particularly problematic for OIF. Tremendous spatial-temporal variability in the stimulated efficiency of the biological pump prevents the simple extrapolation from iron input to carbon sequestration. At best we can attempt to directly measure the induced export flux and infer net sequestration from there. Unfortunately, the dynamic nature of the global ocean not only severely complicates measurements of local export but requires the complete consideration of non-local effects. 

%%%%%%%%%%%%%%
\subsubsection{Challenges measuring local export}

Export production is notoriously difficult to measure. Physical methods such as sediments traps which simply catch particulate “rain” are subject to, amongst other things, grazing by passing zooplankton and hydrodynamic biases over the mouth of the trap \parencite{Buesselerassessmentusesediment2007}. Chemical methods measuring the secular disequilibrium between particle reactive $^{234}Th$ and its conservative, long lived parent radioisotope $^{238}U$ provide a good proxy for export production \parencite{Buesselerdecouplingproductionparticulate1998} but are subject to their own problematic biases and assumptions. By providing multiple lines of evidence EIFEX \parencite{SmetacekDeepcarbonexport2012} was able to convincingly conclude that they induced an increase in local export production, but employing a similarly large suite of measurement tools would hardly be practical in a global scale OIF framework. Anything less, however, may not be reliable.  


%%%%%%%%%%%%%%
\subsubsection{Biogeochemical additionally and establishing a baseline}

Additionality, the notion that offset credits should not be granted for sequestration that would have happened irrespective of a proposed project, is often only considered in an operational context \parencite{LeinenBuildingrelationshipsscientists2008}, but for OIF, a process designed to amplify a natural phenomena, it must also be considered in a biogeochemical context. In order to accurately audit it necessary is to establish a baseline by which to quantify how much additional export has been stimulated beyond what would have occurred naturally. Establishing such a baseline would require nearly continuous control measurements over the entire lifetime of the bloom at multiple locations throughout the surrounding unfertilized waters. Even then, separating the OIF induced signal from high frequency spatial-temporal variability, inter-annual variability and long term climate trends would prove nearly impossible \parencite{CullenPredictingverifyingintended2008}.

EIFEX \parencite{SmetacekDeepcarbonexport2012} cleverly fertilized a water mass trapped in the interior of a large eddy to help control for mixing biases between the fertilized and control patch, but likely introduced new biases as well. Internal eddy dynamics are capable of modifying the in-situ iron flux \parencite{McGillicuddyMechanismsPhysicalBiologicalBiogeochemicalInteraction2016} and accounting for a heightened export flux independent of the stimulus from artificial fertilization.Because the strength and direction of these internal dynamics vary between eddies \parencite{GaubeSatelliteObservationsMesoscale2014} adequately controlling for them would require the impossible task of measuring the same eddy, at the same time, with and without iron fertilization.  

%%%%%%%%%%%%%%
\subsubsection{Spatial/temporal dissonance}

Finally, overcoming the challenges hindering local export measurement may be irrelevant if the local signal does not dominate the net global signal. Ocean circulation and mixing increase spatial scales and distribute the effects of a local perturbation far from it’s point source, severely complicating long term verification and assessment \parencite{BuesselerOceanIronFertilization2008}. Non-local effects, largely triggered by the downstream depletion of macro-nutrients, are thought to be of a similar scale and often in an opposing direction to local effects \parencite{GnanadesikanEffectspatchyocean2003, AumontGlobalizingresultsocean2006, OschliesSideeffectsaccounting2010, SarmientoThreedimensionalsimulationsimpact1991}. Accurate auditing, then, would require estimates of both the locally induced export flux and consideration of all non-local effects \parencite{YoonOceanIronFertilization2016}. Unfortunately, large space and time scales prevent direct measurement of these effects11, while complex nonlinearities prevent reliable model-based estimates for individual deployments.

%%%%%%%%%%%%%%
\subsection{Safety: Will it have adverse side effects?} 

By design, OIF seeks to deliberately manipulate ocean biogeochemistry at the global scale. In a highly complex ocean system it is unreasonable to expect this will not lead to a bevy of broad ranging, unpredictable and unintended consequences. Given the breadth of the climate crisis, the prospect of marginal gains in carbon sequestration may reasonably outweigh the risk of collateral damage, however, there is first an obligation to understand the full scope of potentially harmful side effects and deem them acceptable \parencite{BuesselerOceanIronFertilization2008,CullenPredictingverifyingintended2008}.

%%%%%%%%%%%%%%
\subsubsection{Anoxia and hypoxia} 

If OIF is successful then increased export production will eventually fuel increased aerobic microbial decomposition at depth \parencite{CullenPredictingverifyingintended2008}. Increased microbial decomposition will increasingly consume oxygen and could lead to the development of hypoxia or anoxia below the euphotic zone \parencite{YoonOceanIronFertilization2016}. These deoxygenated subsurface water can eventually be transported to the surface in coastal upwelling systems where they can trigger mass fish die offs \parencite{CullenPredictingverifyingintended2008}.  Similar events have been observed along the Pacific Eastern Boundary current \parencite{GranthamUpwellingdrivennearshorehypoxia2004, ChanEmergenceAnoxiaCalifornia2008}, and are thought to be triggered by non-local anthropogenic nutrient loading.

Simple early box-models predicted that large scale fertilization would create vast subsurface anoxic regions \parencite{SarmientoThreedimensionalsimulationsimpact1991}. Later models countered that oxygen depletion may not be quite as severe, but only because the magnitude of the predicted sequestration flux also decreased \parencite{DenmanClimatechangeocean2008}. Similarly, compensating oxygenation has been predicted to occur at lower latitudes in some models, but only due to a reduction in productivity triggered by upstream nutrient utilization \parencite{OschliesSideeffectsaccounting2010}. The net effect is difficult to constrain, but generally appears qualitatively opposed to the desired outcome of OIF; Net improvement in global export is tied to a net deterioration of subsurface oxygen. 

%%%%%%%%%%%%%%
\subsubsection{Broader ecosystem interactions – productivity, community composition, and fisheries} 

Despite strong evidence of an immediate, local increase in productivity following fertilization, some predict that on decadal timescales OIF will actually lead to a net reduction in global productivity triggered by a reduction in the downstream nutrient supply, particularly to the tropics \parencite{GnanadesikanEffectspatchyocean2003, AumontGlobalizingresultsocean2006,ZaharievPreindustrialhistoricalfertilization2008}. Over long enough time scales a net reduction in primary productivity could ripple up the food web reducing the availability of harvestable fish stocks. \textcite{GnanadesikanEffectspatchyocean2003} estimated that the cost to fisheries could be as high as \$150 per ton of carbon sequestered via OIF. 
The ultimate effect on fisheries is further complicated by the potential for complex, unpredictable changes to ecosystem structure fueled by shifts in species composition at lower trophic levels \parencite{ChisholmWhatControlsPhytoplankton1991}. During sustained fertilization, blooms have been observed to shift to diatom dominance \parencite{MarchettiPhytoplanktonprocessesmesoscale2006}, and in turn favor larger species of zooplankton \parencite{TsudaMesozooplanktonresponseiron2006}. These changes in community composition have at times lead to an increase in the abundance of Pseudo-nitzschia, a diatom genus known to produce the harmful neuro-toxin domoic acid \parencite{TrickIronenrichmentstimulates2010, SilverToxicdiatomsdomoic2010}. 

The precise community response, however, remains largely unpredictable. Even at smaller, experimental scales, ecosystems have been observed to respond differently to multiple fertilizations conducted at the same site \parencite{BoydMesoscaleIronEnrichment2007}. At a global scale these changes could lead to dramatic and unpredictable regime shifts in community composition and more generally regional biogeochemistry \parencite{BoydImpactClimateChange2003}. It is, at best, unclear how major changes in ecosystem structure will effect ocean resources and fisheries.  
%%%%%%%%%%%%%%
\subsubsection{Non-CO2 climate active gasses} 

The net radiative effect of OIF may be significantly altered by modified contributions from non-$CO_2$ climate active gasses, such as nitrous oxide ($N_2O$), methane ($CH_4$), and dimethyl-sulfide ($DMS$).

$N_2O$ is a greenhouse gas roughly 300 times more potent than $CO_2$ on a per-molecule basis \parencite{RamaswamyRadiativeForcingClimate2001}.  Oceanic $N_2O$ production is associated with both the bacterial oxidation of remineralized ammonium to nitrate, as well as the bacterial reminerialization of organic matter at low oxygen levels \parencite{CohenNitrousoxideproduction1979}. The existence of multiple pathways complicates precise estimates of OIF induced $N_2O$ fluxes, but $N_2O$ production is generally thought to increase as increasing export is inevitably decomposed. Observations from the SOIREE iron enrichment experiment \parencite{LawPredictingmonitoringeffects2008} in addition to modelling studies \parencite{JinOffsettingradiativebenefit2003, OschliesSideeffectsaccounting2010} have reported a net increase in $N_2O$ production estimated to compromise the net radiative effect of atmospheric $CO_2$ removal by 5-10\%. EIFEX, one of the largest enrichment experiments to date, however, observed no detectable change in $N_2O$ production \parencite{WalterNitrousoxidemeasurements2005}.

$CH_4$, another greenhouse gas produced during microbial decomposition, is roughly 20 times more potent than $CO_2$ \parencite{RamaswamyRadiativeForcingClimate2001}. Oceanic $CH_4$ is produced predominately by bacteria in completely anoxic microhabitats associated with sinking particulate organic matter \parencite{KarlProductiontransportmethane1994}. While increased export would increase the prevalence of these microhabitats, the net potential for OIF induced $CH_4$ production to offset atmospheric $CO_2$ reductions is not expected to exceed 1\% \parencite{OschliesSideeffectsaccounting2010}. 

$DMS$, unlike $N_2O$ and $CH_4$, is not a greenhouse gas. Instead, $DMS$ leads to the creation of sulfur aerosols which in turn help seed cloud formation, working to cool the atmosphere by increasing earth’s albedo. Oceanic $DMS$ is produced as byproduct by marine phytoplankton and has been proposed as a biological pathway for climate regulation \parencite{CharlsonOceanicphytoplanktonatmospheric1987}. While the nature of this regulatory loop has been found substantially more complex than initially hypothesized \parencite{AyresAtmosphericsulphurcloud1997, Quinncaseclimateregulation2011}, it remains a significant link between marine biota and climate. The net effect of OIF on DMS production, however, is unclear. Some enrichment experiments have seen an increase in $DMS$ production immediately after fertilization \parencite{TurnerIncreaseddimethylsulphide1996,TurnerIroninducedchangesoceanic2004, LissOceanfertilizationiron2005}, with hikes as high as 6.5-fold \parencite{TurnerIroninducedchangesoceanic2004}. Other experiments, however, have reported no change in $DMS$ production \parencite{Takedasituironenrichmentexperiment2005,NagaoResponsesDMSseawater2009}, or even observed a decrease following an initial spike \parencite{LevasseurDMSPDMSdynamics2006}. 

Ultimately, without observations of patch-scale fertilizations longer than 1-2 months, no less for a continuous, basin wide fertilization program, the forcing on non-$CO_2$ climate active gasses remains largely unknown \parencite{LawPredictingmonitoringeffects2008}. Constraining these fluxes is critical to understanding the net radiative effect of OIF and will require a better understanding of changes to community composition, particulate export, and deep bacterial remineralization at time and space scales much larger than a single fertilization.  

%%%%%%%%%%%%%%
\subsubsection{Ocean acidification} 

Finally, the desired uptake of $CO_2$ into ocean will only exacerbate ocean acidification. Ocean acidification is caused as increasing $CO_2$ shifts the equilibrium of the carbonate system in favor of an increasing concentration of $H+$ ions, thus reducing the $pH$. Ocean acidification has been widely shown to be detrimental to some marine biota, particularly calcifiers \parencite{DoneyOceanAcidificationOther2009}. While the majority of anthropogenic $CO_2$ may inevitably end up in the ocean regardless, if successful, OIF will undoubtedly increase the rate at which it is added, giving organisms less time to adapt \parencite{DenmanClimatechangeocean2008}.  


%%%%%%%%%%%%%%
\subsection{Managing uncertainty and market failures} 

Establishing any chance for a successful global scale OIF operation would call for careful consideration into when and where each individual fertilization is implemented \parencite{YoonOceanIronFertilization2016, BuesselerOceanIronFertilization2008}. Collectively optimizing a global portfolio of fertilization sites would require intensive monitoring coupled to a comprehensive adaptive management plan. Monitoring would need to account for local and non-local sequestration, a baseline by which to establish additionality, and a suite of complex side effects. Scientific decision making would need to be equipped to understand and react to unpredictable developments on time and space scales well beyond the scope of local fertilization \parencite{GnanadesikanEffectspatchyocean2003}. Implementation would need to be flexible enough to change course on the fly, but also incremental enough to safeguard against significantly time-lagged, down stream consequences. 

Given the challenges associated with accurate measurement, developing such a demanding management program would be exceedingly difficult, even under unilateral authority. It might be impossible under market control. Institutional inefficiencies would dramatically hinder the ability of myriad independent private corporations to cohesively implement a complex and dynamic plan. At best, any practical approach to establishing a baseline and quantifying additionality would have to be done in a broad mean sense. This sort of generalization creates the exact problem that plagues other offset markets \parencite{GillenwaterWhatadditionalityPart2012}. Corporations looking to maximize the differential between the baseline and outcome will be incentivized to seek underestimated baselines rather than improved outcomes. Even well intentioned incentives designed to ensure corporations fertilize the right places at the right times could be corrupted by challenges to enforcement and unreliable auditing. Finally, no project would be truly oceanographically independent, making it impossible to appropriately distribute the costs and benefits across operationally independent projects.

Ultimately, it is difficult to see how market forces could ensure OIF conforms to the best scientific judgment. Consequently, these market failures will compound the already substantial risk of adverse side effects and damp our ability to react to emerging threats, all while introducing problematic economic and legal ramifications. Economically, without a robust auditing framework to ensure fair and consistent compensation, corporations incentivized to game the system could undermine the entire global offset markets. Legally, if harmful consequences do arise, establishing liability will be nearly impossible across complex, multi-actor, non-local, time-lagged lines of etiology. 
Taken together, the inability of markets to properly manage the uncertainty and risk associated with OIF on a global scale or resolve the challenges imposed by tremendous spatial-temporal dissonance highlight the case against commercialization.

%%%%%%%%%%%%%%
\section{The case for continued research}

While few scientists support commercialization, a much more controversial debate has ensued over the prudence of continued research into OIF. Given the tremendous concerns over commercialization many argue that further research would only misallocate finite scientific resources and bolster a moral hazard threatening to distract from more legitimate mitigation efforts. 

On the other hand, the case for continued research is three-fold. First, from a basic research perspective, unraveling the role of iron in our oceans and its contribution to glacial-interglacial climate variability will help shape our understanding of ocean biogeochemistry and climate change. Second, it is not impossible that some realization of OIF could eventually be safely and thoughtfully implemented under unilateral governmental authority as one of many useful tools to address climate change, particularly if mitigation efforts continue to fail. Third, if there is reason to believe that commercialization could proceed contrary the scientific consensus, then continued research could help deter reckless behavior by delegitimizing unfounded commercial deployments. Argument three warrants further consideration, and in Section 3.1 I outline the incentives that could drive development on voluntary markets and the role that continued research can play moving forward.

%%%%%%%%%%%%%%
\subsection{Prospects for commercialization on voluntary offset markets}

It is unlikely that any global governance framework would blatantly disregard scientific wisdom and begin granting offset credits for OIF under heavily regulated compliance offset markets (COMs), however, there are no such barriers to entry on voluntary offset markets (VOMs). VOMs differ from COMs, such as those implemented by the Kyoto Protocol, in that offsets are bought and sold without any federally mandated obligation. Trading on VOMs is instead motivated by a sense a of personal responsibly, corporate branding or an expectation of impending regulations. Lacking significant oversight, the VOMs could serve as a vital seed ground for the development of private OIF ventures. 

%%%%%%%%%%%%%%
\subsubsection{Size of the voluntary market}

Compared to the \$50-100 billion total value of COMs \parencite{CarbonMarketMonitor2016}, VOMs pale in size. Still, with 63.4 $MtCo_2e$ traded in 2016 for a total of $\$191.3$ million \parencite{HamrickUnlockingPotentialState2017}, there is considerable room for a small company to secure lucrative profits. In 2016, the average price for all transactions was $\$3.0/tCO_2e$ but ranged wildly from $\$0.5/tCO_2e$ to as high as $\$50.0/tCO_2e$. Despite some volatility in market size (demand rose by 10\% in 2015 \parencite{HamrickRaisingAmbitionState2016} but dropped off by 24\% in 2016 \parencite{HamrickUnlockingPotentialState2017}) and the growth of COMs in the wake of COP 21, there is an expectation that VOMs will remain viable with the overall demand for sustainable development and industry interest in carbon neutrality is on the rise \parencite{HamrickUnlockingPotentialState2017}. 
%%%%%%%%%%%%%%
\subsubsection{Perception of low cost alternative}

Relative to other carbon offset projects, there is a perception that OIF is a substantially cheaper, economically viable solution \parencite{KeithClimateStrategyCo22006,WorstallCheapWayDeal2012}. Given the low cost of iron and the very high stoichiometric ratio with which $CO_2$ is fixed and iron utilized, it is easy to understand how OIF can appear, at first glance, seductively affordable. Early projections estimated the cost of OIF as low as $\$1-2/tCO_2e$ \parencite{MarkelsSequestrationCarbonDioxide2002}, leaving substantial room to profit over the average price currently being traded on VOMs.

Of course, the reality may be much less enticing. Early estimates were biased by problematic assumptions regarding export efficiency, downstream macro-nutrient depletion and $N_2O$/$CH_4$ compensatory fluxes, and did not internalize the true cost of research and development, monitoring, or delivery systems at the global scale \parencite{WatsonDesigningnextgeneration2008}. Revised projections generally range from $\$8-80/tCO_2e$ \parencite{BoydImplicationslargescaleiron2008}, but still typically exclude the price of potentially harmful externalities which have been estimated as high as $\$150/tCO_2e$ to fisheries alone \parencite{GnanadesikanEffectspatchyocean2003}. One, more extreme, estimate argues for an almost certainly prohibitory price of $\$457/tCO_2e$ \parencite{Harrisonmethodestimatingcost2013}. Perhaps most accurately though, in light of the vast uncertainty in what we can infer from patch scale fertilizations and model integrations, the truth is that we just do not know how expensive long term iron induced sequestration might be at a global scale \parencite{BarkerChapter11Mitigation2007}.

What we do know, however, is that it is not very expensive to dump iron into the ocean. 
Even if the cost of sequestration in a comprehensive, credible OIF scheme is prohibitory, the operational startup cost for small-scale, speculative operators is not. Whether these deployments could ever actually sequester what they claim remains uncertain, but it is precisely this uncertainty that creates the opportunity to discount the risk portfolio and overvalue the chance to win big. More research is needed to constrain the true cost of sequestration and delegitimize the idea that it can be easily extrapolated from one-off, patch-scale, fertilization experiments. 

%%%%%%%%%%%%%%
\subsubsection{Investing in the future and establishing intellectual property}

Without a firm grasp of the true cost of sequestration and buttressed by the low price of iron and overhead, there is a reasonable economic argument for small-scale operators to invest in the development of OIF. Even if the odds for success are slim, the stakes are low (economically if not environmentally) and the jackpot is huge. In the most optimistic, albeit unlikely, scenarios, OIF could generate billions of dollars worth of offsets \parencite{CullenPredictingverifyingintended2008}, fundamentally disrupting carbon markets \parencite{NeeffMarketMethodologiesMeet2007}. Relative to the low cost of investment, the upside is high enough that even if OIF is not immediately profitable it could be justified as a shrewd investment. 

It is exactly these sort of financially low risk, high reward ventures that appear poised for success on VOMs, which are seen as promisingly fertile soil to test emerging carbon sequestration technologies \parencite{HamrickUnlockingPotentialState2017, HamrickRaisingAmbitionState2016}. In the hope that one day OIF will be adopted into compliance markets, it is relatively affordable for entrepreneurs to stage preliminary development on VOMs with an eye to test methodologies and preemptively establish intellectual property. If further research can clarify the biogeochemical impediments to a market-based OIF approach and diminish the perception that it will ever be viable on COMs, it will deter speculative investment on VOMs. 


%%%%%%%%%%%%%%
\subsubsection{Regulatory and legal framework}

Exploratory development on VOMs benefits from a loose regulatory environment with no federally mandated oversight. While a suite of standardization bodies have emerged to administer credibility by verifying that projects are, in fact, permanent, additional, and free of leakage \parencite{HamrickUnlockingPotentialState2017}, it is unclear how a collection of independent auditors would cohesively overcome immense monitoring challenges to develop a comprehensive OIF validation scheme. It is more likely that commercial operators would shop between diverse auditing options before settling on the most economically favorable framework. Even if more reputable standards bodies refuse to accredit OIF, there is nothing blocking the emergence of new organizations willing to do so, and only the scientific community would be equipped to discredit them. This regulatory flexibility relaxes the burden of proof that should be expected from commercial operators and could encourage reckless development.  

In theory, the United Nations Convention on the Law of the Seas and the London Convention, which generally ban dumping on the high seas, should prevent overtly harmful operations. Unfortunately when if comes to OIF, international jurisdiction is often vague and difficult to enforce \parencite{BertramPotentialOceanIron2011}. In 2008, the London Protocol, with support from the International Maritime Organization \parencite{IMONoteInternationalMaritime2008}, banned commercial fertilization \parencite{ResolutionLCLPRegulation2008}, but ambiguity arose over what constituted a permissible, legitimate scientific activity\parencite{BertramPotentialOceanIron2011, GoodellLittleCashSide2011}. In 2013, the UN formally recognized OIF as geoengineering and mandated stricter environmental assessments and approval prior to permitting \parencite{ResolutionLPAmendment2013}, but questions remain over the capacity to enforce international law in the remote Southern Ocean without any centralized legal authority. Poor enforcement and protections under the guise of science have long sheltered the whaling industry \parencite{MangelWhalessciencescientific2016} and could potentially do the same for OIF. Without legitimate research programs for reference, it could be even easier for commercial operations to feign scientific legitimacy.  

%%%%%%%%%%%%%%
\subsubsection{Ethical arguments and public perception}

OIF can only leverage a loose regulatory and legal framework if there is a market to support it. The question of whether there is demand for legally and scientifically suspect offset credits, however, might not be a strictly economic one. Absent any formal obligations, buyers on VOMs are often driven by an earnest desire to do what is right. Given the increasing threat of climate change and the deteriorating state of ocean health, it is not difficult to see how OIF could be sold to a less informed public as an ethically viable gamble.  

By fixating only on favorable outcomes, OIF can and has been spun as marine forestation \parencite{GoodellLittleCashSide2011a}; A global gardening project to support ecosystem health, boost global fisheries, and help feed the world, all while sequestering billions of tons of carbon. Buyers are increasingly interested in these sort of social and environmental co-benefits and are willing to pay for them \parencite{HamrickRaisingAmbitionState2016}. Terrestrial analogs like forestry and land use projects promise similar protection to ecosystem services and successfully captured the second largest market share at the highest average trading price on VOMs in 2016. Together, they were three times more valuable than renewables \parencite{HamrickRaisingAmbitionState2016}. Current scientific wisdom suggests this comparison is not justified, but without continued research, commercial operators may be able to leverage lingering uncertainty to cast OIF in a positive light and shape public opinion, and in a buyers market, where there is no obligation to participate, public perception is tantamount to value.  

%%%%%%%%%%%%%%
\subsubsection{Early Case Studies}

By the mid-late 2000s startups championing OIF had gained considerable momentum. Planktons, founded at the turn of the century by Russ George with an expressed intent to “save the world and make a little cash on the side”, had acquired an oceanographic vessel \parencite{GoodellLittleCashSide2011b}. GreenSea Ventures, a Virginia based outfit, had begun to establish intellectual property by securing patents for several iron delivery strategies \parencite{BowiePositionanalysisocean2016}. Most notably, Climos, had acquired 3.5 million dollars in series A venture capital funding and the support of Elon Musk \parencite{NewPlanktonSeedingVenture2008}. The reputability of these organizations ranged from a pirate-like disregard for scientific nuance at Planktos \parencite{GoodellLittleCashSide2011b}, to an active engagement with the scientific community at Climos \parencite{LeinenBuildingrelationshipsscientists2008}.

By the end of the decade, however, the scientific consensus had begun to crystalize; It was too early to support the commercialization of OIF \parencite{BuesselerOceanIronFertilization2008}. With the increasingly negative perception coupled to legal challenged introduced by the London Protocol \parencite{ResolutionLCLPRegulation2008}, the tide began turn for the first wave of OIF startups. In 2008 Planktos was forced to halt operations mid deployment after investors jumped ship, and shutdown all together shortly there after. Climos, lasted longer but ultimately could not withstand the legal and scientific scrutiny, eventually shifting their focus towards broader geoengineering technologies \parencite{GoodellLittleCashSide2011c} before appearing to disappear completely.  

Nevertheless, the threat of commercial development has not receded entirely. In 2012 Russ George reemerged to dump some 100 tons of iron sulphate into the Pacific Ocean \parencite{LukacsWorldbiggestgeoengineering2012}. More recently, in 2017, the Oceaneos Marine Research Foundation sparked controversy when it began to seek permits to dump 10 tons of iron off the Chilean coast \parencite{TollefsonIronDumpingOceanExperiment2017}.  If long term mitigation efforts continue to fail, it is unlikely that commercial interest in OIF will disappear on its own. Further research, is needed to continue to unravel the complexity, uncertainty and risk inherent in a global scale OIF platform and highlight the challenges to market based management the should preclude commercialization from consideration.  


%%%%%%%%%%%%%%
\section{Conclusions}

The prospects for a successful, basin wide, OIF campaign are shrouded in uncertainty and appear dubious at best. Even if iron is able to stimulate productivity at scale in HNLCs it is unclear how much will be exported to depth to be sequestered for adequate time scales and to what degree complex, non-linear, feedbacks into ecosystem structure and downstream macro-nutrient utilization might compromise net atmospheric $CO_2$ drawdown. Worse, we don’t fully understand how deliberately manipulating ocean biogeochemistry at this scale could threaten intricate marine ecosystems, global fisheries, or the flux of non-$CO_2$ climate active gases.

Nevertheless, the climate crisis is not going to disappear and all potential contributions to a solution warrant thorough consideration. Risk and uncertainty will weigh heavily in all options and must not alone preclude any from consideration. Success, then, will be largely predicated on selecting strategies that can optimize our ability to manage uncertainty and improve our preparation to adapt to unpredictable developments along the way. It is resoundingly clear that a market-based approach to OIF will do neither. 

As continual fertilization induces an increasingly non-local and time-lagged response across the ocean it will become impossible to accurately distribute responsibility across many independent operations. The inability to attribute the true and total consequences of any individual fertilization, compounded by the uncertainty associated with measuring even the net effect of all integrated deployments, derails any hope for reliable auditing. Without a robust auditing framework or serviceable enforcement, misaligned incentives will prevent market forces from aligning with a rapidly evolving fertilization scheme and compromise our ability to react and adapt to unpredictable developments beneath a cloud of uncertainty.

If OIF was ever to be implemented, likely as a last ditch effort, it should be under unilateral authority and with dynamic management. Continued, incremental, research will not only clarify the viability of a potential emergency deployment, but is, perhaps more importantly, critical in deterring the reckless development of OIF on VOMs. Without consumer confidence that offsets are meaningful, there is no financial or ethical imperative to drive demand on VOMs. As it stands, the uncertainty surrounding OIF is large enough to justify a diversity of public opinion, particularly if developers highlight only positive outcomes. Further research is needed to reduce uncertainty and constrain public perception by continuing to clarify the risks, elaborate the challenges, and delegitimize the promise of a silver bullet.
